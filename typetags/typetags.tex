\documentclass[runningheads]{llncs}

\usepackage{makeidx}  % allows for indexgeneration
\usepackage{url}
\usepackage[xetex]{hyperref}
\usepackage{todonotes}
\usepackage{listings}
\usepackage{fontspec}
\usepackage{fancyvrb}
\usepackage{mathpartir}
\usepackage[labelfont=bf]{caption}
\usepackage{subcaption}
\captionsetup{compatibility=false}
\usepackage[bottom]{footmisc}
\usepackage{longtable}
\usepackage{array}
\usepackage{booktabs}
\usepackage{colortbl}%
  \newcommand{\myrowcolour}{\rowcolor[gray]{0.925}}
\usepackage{wasysym}

\usepackage{microtype}
\usepackage{times}
\usepackage[english]{babel}
\usepackage{graphicx}
\usepackage{xcolor}
\usepackage{lipsum}

\newcommand{\comment}[1]{}

% Member sequences
\newcommand{\seq}[1]{\overline{#1}}

% arrays
\newcommand{\ba}{\begin{array}}
\newcommand{\ea}{\end{array}}
\newcommand{\bda}{\[\ba}
\newcommand{\eda}{\ea\]}
\newcommand{\ei}{\end{array}}
\newcommand{\bcases}{\left\{\begin{array}{ll}}
\newcommand{\ecases}{\end{array}\right.}
\newcommand{\sporeurl}{\url{https://github.com/scala/spores}}
% spacing
\newcommand{\gap}{\quad\quad}
\newcommand{\andalso}{\quad\quad}
\newcommand{\biggap}{\quad\quad\quad}
\newcommand{\nextline}{\\ \\}
\newcommand{\htabwidth}{0.5cm}
\newcommand{\tabwidth}{1cm}
\newcommand{\htab}{\hspace{\htabwidth}}
\newcommand{\tab}{\hspace{\tabwidth}}
\newcommand{\linesep}{\ \hrulefill \ \smallskip}

\newcommand{\ie}{{\em i.e.,~}}
\newcommand{\eg}{{\em e.g.,~}}
\newcommand{\etc}{{\em etc}}

\newcommand*\loc{\includegraphics[height=0.65em,keepaspectratio]{loc}}
\newcommand*\stars{\includegraphics[height=0.8em,keepaspectratio]{stars}}
\newcommand*\contribs{\includegraphics[height=0.8em,keepaspectratio]{contribs}}

\lstdefinelanguage{Scala}%
{morekeywords={abstract,case,catch,char,class,%
    def,else,extends,final,%
    if,import,%
    match,module,new,null,object,override,package,private,protected,%
    public,return,super,this,throw,trait,try,type,val,var,with,implicit,%
    macro,sealed,%
  },%
  sensitive,%
  morecomment=[l]//,%
  morecomment=[s]{/*}{*/},%
  morestring=[b]",%
  morestring=[b]',%
  showstringspaces=false%
}[keywords,comments,strings]%

% \lstset{language=Scala,%
%   mathescape=true,%
%   columns=[c]fixed,%
%   numbers=left, numberstyle=\scriptsize\color{gray}\ttfamily,
%   basewidth={0.5em, 0.40em},%
%   basicstyle=\tt,%
%   xleftmargin=0.0cm
% }

\setmainfont[
  Ligatures=TeX,
  SmallCapsFont={TeX Gyre Termes},
  SmallCapsFeatures={Letters=SmallCaps},
]{Times New Roman}

\lstset{tabsize=2,
basicstyle=\ttfamily\fontsize{9pt}{1em}\selectfont,
commentstyle=\itshape\rmfamily,
numbers=left, numberstyle=\scriptsize\color{gray}\ttfamily, language=scala,moredelim=[il][\sffamily]{?},mathescape=false,showspaces=false,showstringspaces=false,xleftmargin=15pt,escapechar=@, morekeywords=[1]{let,fn,val},deletekeywords={for},classoffset=0,belowskip=\smallskipamount
}

% \setlength{\belowcaptionskip}{-5pt}
\setlength{\textwidth}{12.4cm}
 \setlength\intextsep{ 0pt plus 2pt minus 2pt}

% \titlespacing{\section}{0pt}{\parskip}{-\parskip}

\begin{document}

% \fontspec[
%   SmallCapsFont={TeX Gyre Termes},
%   SmallCapsFeatures={Letters=SmallCaps},
% ]{Times New Roman}
\VerbatimFootnotes
\setmonofont[Scale=0.8,BoldFont={Consolas Bold}]{Consolas}

%
\mainmatter              % start of the contributions

% Title ideas:

\title{A New Design for Runtime Type Tags in Scala}
% \subtitle{A Foundation for Lambdas in the Age of Concurrency and Distribution}
\titlerunning{\hspace{-0.9cm} A New Design for Runtime Type Tags in Scala}  % abbreviated title (for running head)
%                                     also used for the TOC unless
%                                     \toctitle is used
%
\author{Philipp Haller$^{1}$ \and Heather Miller}

%%% list of authors for the TOC (use if author list has to be modified)
\tocauthor{Philipp Haller$^{1}$ and Heather Miller}

% \institute{EPFL and Typesafe, Inc$^{1}$\\

% \texttt{\scriptsize \{heather.miller, martin.odersky\}@epfl.ch}
% and \texttt{\scriptsize philipp.haller@typesafe.com}$^{1}$}


\institute{EPFL and Typesafe, Inc.$^{1}$\\
\texttt{philipp.haller@typesafe.com}$^{1}$
and
\texttt{heather.miller@epfl.ch}}

% \institute{EPFL, Switzerland\\
% \and
% Typesafe, Switzerland}

\setlength{\abovecaptionskip}{0pt}
\setlength{\belowcaptionskip}{0pt}
\authorrunning{\hspace{-3cm} P. Haller and H. Miller} % abbreviated author list (for running head)

\maketitle              % typeset the title of the contribution

\begin{sloppypar}
\begin{abstract}

Runtime type representations are a powerful tool for generic programming.
Scala provides built-in language support for two varieties of type tags: type
tags representing erased types, and type tags representing non-erased types.
The latter form is currently experimental. While type tags in Scala already
provide several desirable features, their current design is limited: type tags
are not serializable, which fundamentally restricts their use for distributed
programming; and type tags are not supported on all existing compiler back-
ends. In particular, type tags cannot be supported on Scala's experimental
JavaScript back-end. This paper presents a new design for type tags in Scala,
which addresses both challenges: in the new design type tags are serializable
and portable for use on both the JVM and the JavaScript backend. We evaluate
our design in the context of Scala/Pickling, an industrial-strength
serialization framework for Scala.

% == 150-word abstract for the submission page ==


\keywords{runtime types, serialization}
\end{abstract}
%
\vspace{-3mm}
\section{Introduction}

\section{Type tags in Scala}

Type tags in Scala provide runtime type representations for types. These
runtime representations are related to the class \verb|java.lang.Class<T>| of
the JDK.

Type tags are automatically generated by the Scala compiler. Generation is
triggered using Scala's implicits.

Methods in Scala can have implicit parameter lists...

If the type of an implicit parameter has the shape \verb|TypeTag[T]|, instead
of simply looking up an implicit value, the compiler {\em automatically
generates} a type tag for type \verb|T|.


\subsection{ClassTag, TypeTag and WeakTypeTag}

\section{Conclusion}
\label{sec:conclusion}
\vspace{-1mm}

We presented a new design for Scala's type tags.

\bibliographystyle{abbrv}
\bibliography{bib}

\end{sloppypar}
\end{document}
